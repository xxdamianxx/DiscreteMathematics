\documentclass[11pt]{article}

\usepackage{fancyhdr}
\usepackage[letterpaper]{geometry}
\usepackage[latin2]{inputenc}
\usepackage{graphicx}
\usepackage{ulem}
\usepackage{amsmath}

\pagestyle{fancyplain}
\lhead{\large{CSE 015: Discrete Mathematics}}
\rhead{\large{Fall Semester 2017}}

\title{\bf Homework Assignment 6}
\date{Sunday, December 10 2017}
\author{Pedro Damian Sanchez Jr}

\begin{document}

\maketitle

\section*{Question 1}

There are 18 mathematics majors and 325 computer science majors at a college.

\begin{itemize}
\item In how many ways can two representatives be picked so that one is a mathematics major and the other is a computer science major?
\item In how many ways can one representative be picked who is either a mathematics major or a computer science major?
\end{itemize}

\section*{Question 1 Solution}

\begin{itemize}
\item Since there are 18 math majors, there are 18 ways to choose a math major, and, since there are 325 computer science majors, there are 325 ways to choose a computer science major. Therefore, there are $18*325=5850$ ways to select a representative for each group.
\item Since only one person needs to be selected to represent both groups, and, there is no regard as to which pool the representative is chosen from, there are $18+325=343$ ways to select one representative.
\end{itemize}

\section*{Question 5}

Six different airlines fly from New York to Denver and seven fly from Denver to San Francisco. How many different pairs of airlines can you choose on which to book a trip from NewYork to San Francisco via Denver, when you pick an airline for the flight to Denver and an airline for the continuation flight to San Francisco?

\section*{Question 5 Solution}

It is known that there are six airlines which travel from New York to Denver, and, seven airlines which travel from Denver to San Francisco. Therefore, if a traveller wants to fly from New York to San Francisco, the total possible arrangements transversing the Denver airport comes out to $6*7=42$ different combinations in flight options.

\section*{Question 7}

How many different three-letter initials can people have?

\section*{Question 7 Solution}

Since there are twenty six letters in the english alphabet, each inital has a total of twenty six possibilities. Therefore, there are a total of $26*26*26=26^3=17,576$ possible combinations of three-letter initials.

\section*{Question 8}

How many different three-letter initials with none of the letters repeated can people have?

\section*{Question 8 Solution}

There exists a total of $26*25*24=15,600$ possible combinations of three-letter initials.

\section*{Question 9}

How many different three-letter initials are there that begin with an A?

\section*{Question 9 Solution}

There exists a total of $26*26=26^2=676$ possible combinations of three-letter initials beginning with the letter A.

\section*{Question 14}

How many bit strings of length $n$, where $n$ is a positive integer, start and end with 1's?

\section*{Question 14 Solution}

The number of bit strings with length $n$ is $2^n$. However, if the first and last positions of these strings are restricted to consist of only 1's, then the number of possible arrangements falls to $2^{n-2}$.

\section*{Question 46}

In how many ways can a photographer at a wedding arrange 6 people in a row from a group of 10 people, where the bride and the groom are among these 10 people, if:

\begin{itemize}
\item the bride must be in the picture?
\item both the bride and groom must be in the picture?
\item exactly one of the bride and the groom is in the picture?
\end{itemize}

\section*{Question 46 Solution}

\begin{itemize}
\item If the bride is to be in the picture then she must always be one of the 6 in the arrangement, which leaves the other 9 peope to be positoned in any assortment. This gives us $6*9*8*7*6*5=90,720$ possible way to arrange a group photo with the lovely bride.
\item If both the bride and the groom are always to appear in the photo shot then there are $6*5*8*7*6*5=50,400$.
\item Same shit as the first one, but now with both bride only and groom only, so it's got to look like this $2*6*8*7*6*5*4=80,640$.
\end{itemize}

\end{document}