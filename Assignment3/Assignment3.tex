\documentclass[11pt]{article}

\usepackage{fancyhdr}
\usepackage[letterpaper]{geometry}
\usepackage{graphicx}


\pagestyle{fancyplain}
\lhead{\large{CSE 015: Discrete Mathematics}}
\rhead{\large{Fall Semester 2017}}


\title{\bf Homework Assignment 3}
\date{Sunday, October 8 2017}
\author{ Pedro Damian Sanchez Jr}

\begin{document}

\maketitle

\section{Section-2.1 Sets}

\subsection{Problem 1}

List the members of these sets.
\begin{itemize}
\item a) [$x | x$ is a real number such that $x^2 = 1$] \\
\item b) [$x | x$ is a positive integer less than 12] \\
\item c) [$x | x$ is the square of an integer and $x < 100$] \\
\item d) [$x | x$ is an integer such that $x^2 = 2$] \\
\end{itemize}

\subsection{Problem 1 Solution}

The Member Sets are as follows:
\begin{itemize}
\item a) $x = [-1, 1]$ \\
\item b) $x = [1, 2, 3, 4, 5, 6, 7, 8, 9, 10, 11]$ \\
\item c) $x = [0, 1, 4, 9, 16, 25, 36, 49, 64, 81]$ \\
\item d) no solution \\
\end{itemize}

\subsection{Problem 2}

Use set builder notation to give a description of each of these sets.

\begin{itemize}
\item a) [0, 3, 6, 9, 12] \\
\item b) [−3,−2,−1, 0, 1, 2, 3] \\
\item c) [m, n, o, p] \\
\end{itemize}

\subsection{Problem 2 Solution}

Set Builder Notation solutions are as follows:

\begin{itemize}
\item a) \\
\item b) \\
\item c) \\
\end{itemize}

\subsection{Problem 5}

Determine whether each of these pairs of sets are equal.

\begin{itemize}
\item a) [1, 3, 3, 3, 5, 5, 5, 5, 5] and [5, 3, 1] \\
\item b) [[1]] and [1, [1]] \\
\item c) [$\emptyset$] and $\emptyset$ \\
\end{itemize}

\subsection{Problem 5 Solution}

\begin{itemize}
\item a) [1, 3, 3, 3, 5, 5, 5, 5, 5] and [5, 3, 1] are equal. \\
\item b) [[1]] and [1, [1]] are not equal. \\
\item c) [$\emptyset$] and $\emptyset$ are not equal. \\
\end{itemize}

\subsection{Problem 6}

Suppose that A = [2, 4, 6], B = [2, 6], C = [4, 6], and D = [4, 6, 8]. Determine which of these sets are subsets of which other of these sets.

\subsection{Problem 6 Solution}

 B = [2, 6] is a subset of A = [2, 4, 6] \\
\\
 C = [4, 6] is a subset of A = [2, 4, 6], and, D = [4, 6, 8] \\

\subsection{Problem 7}

For each of the following sets, determine whether 2 is an element of that set.

\begin{itemize}
\item a) [$x \in R | x$ is an integer greater than 1]
\item b) [$x \in R | x$ is the square of an integer]
\item c) [2,[2]]
\item d) [[2],[[2]]]
\item e) [[2],[2,[2]]]
\item f ) [[[2]]]
\end{itemize}

\subsection{Problem 7 Solution}

\begin{itemize}
\item a) $1 < 2$, therefore 2 is an element of the set.
\item b) The square root of 2 is not an integer, therefore 2 is not an element of the set.
\item c) 2 is an element of the set.
\item d) 2 is not an element of the set.
\item e) 2 is not an element of the set.
\item f ) 2 is not an element of the set.
\end{itemize}

\subsection{Problem 11}

Determine whether each of these statements is true or false.

\begin{itemize}
\item a) $x \in [x]$
\item b) $[x] \subset [x]$
\item c) $[x] \in [x]$
\item d) $[x] \in [[x]]$
\item e) $\emptyset \subset  [x]$
\item f ) $\emptyset \in [x]$
\end{itemize}

\subsection {Problem 11 Solution}

\begin{itemize}
\item a) $x \in [x]$ is true, $x$ is an element within the set.
\item b) $[x] \subset [x]$ is true, every set is a subset within itself.
\item c) $[x] \in [x]$ is false, a set can not belong to itself.
\item d) $[x] \in [[x]]$ is true, as the set exists within the set.
\item e) $\emptyset \subset  [x]$ is true, every set posseses within it an empty set.
\item f ) $\emptyset \in [x]$ is false, an empty set can not be considered an element of a set.
\end{itemize}

\subsection{Problem 12}

Use a Venn Diagram to illustrate the subset of odd integers in the set of all positive integers not exceeding 10.

\subsection{Problem 12 Solution}

\begin{itemize}
\item a) \\
\\
\\
\\
\\
\\
\end{itemize}

\subsection{Problem 14}

Use a Venn diagram to illustrate the relationship A $\subset$  B and B $\subset$ C.

\subsection{Problem 14 Solution}

\begin{itemize}
\item a) \\
\\
\\
\\
\\
\\
\end{itemize}

\subsection{Problem 18}

Find two sets A and B such that A $\in$ B and A $\subset$ B.

\subsection{Problem 18 Solution}

A = $\emptyset$ \\
\\
If B = [$\emptyset$, [$\emptyset$]], then A $\in$ B and A $\subset$ B \\

\subsection{Problem 20}

What is the cardinality of each of these sets?

\begin{itemize}
\item a) $\emptyset$
\item b) [$\emptyset$]
\item c) [$\emptyset$, [$\emptyset$]]
\item d)[$\emptyset$, [$\emptyset$], [$\emptyset$], [$\emptyset$]]]
\end{itemize}

\subsection{Problem 20 Solution}

\begin{itemize}
\item a) The cardinality of $\emptyset$ is $|\emptyset| = 0$
\item b) The cardinality of $[\emptyset]$ is $|[\emptyset]| = 1$ 
\item c) The cardinality of $[\emptyset, [\emptyset]]$ is $|[\emptyset, [\emptyset]]| = 2$
\item d) The cardinality of $[\emptyset, [\emptyset], [\emptyset, [\emptyset]]]$ is $|[\emptyset, [\emptyset], [\emptyset, [\emptyset]]]| = 3$
\end{itemize}

\subsection{Problem 21}

Find the power set of each of these sets, where $a$ and $b$ are distinct elements.

\begin{itemize}
\item a) $[a]$
\item b) $[a, b]$
\item c) $[\emptyset, [\emptyset]]$
\end{itemize}

\subsection{Problem 21 Solution}

\begin{itemize}
\item a) The Power Set of $[a]$ is $[\emptyset, [a]]$
\item b) The Power Set of $[a, b]$ is $[\emptyset, [a], [b], [a,b]]$
\item c) The Power Set of $[\emptyset, [\emptyset]]$ is $[\emptyset, [\emptyset], [[\emptyset]], [\emptyset, [\emptyset]]]$
\end{itemize}

\subsection{Problem 27}

Let A = $[a, b, c, d]$ and B = $[y, z]$; Find:

\begin{itemize}
\item a) A x B
\item b) B x A
\end{itemize}

\subsection{Problem 27 Solution}

\begin{itemize}
\item a) A x B = $[(a, y), (b, y), (c, y), (d, y), (a, z), (b, z), (c, z), (d, z)]$
\item b) B x A = $[(y, a), (y, b), (y, c), (y, d), (z, a), (z, b), (z, c), (z, d)]$
\end{itemize}

\end{document}